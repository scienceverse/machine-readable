\PassOptionsToPackage{unicode=true}{hyperref} % options for packages loaded elsewhere
\PassOptionsToPackage{hyphens}{url}
%
\documentclass[english,doc,floatsintext]{apa6}
\usepackage{lmodern}
\usepackage{amssymb,amsmath}
\usepackage{ifxetex,ifluatex}
\usepackage{fixltx2e} % provides \textsubscript
\ifnum 0\ifxetex 1\fi\ifluatex 1\fi=0 % if pdftex
  \usepackage[T1]{fontenc}
  \usepackage[utf8]{inputenc}
  \usepackage{textcomp} % provides euro and other symbols
\else % if luatex or xelatex
  \usepackage{unicode-math}
  \defaultfontfeatures{Ligatures=TeX,Scale=MatchLowercase}
\fi
% use upquote if available, for straight quotes in verbatim environments
\IfFileExists{upquote.sty}{\usepackage{upquote}}{}
% use microtype if available
\IfFileExists{microtype.sty}{%
\usepackage[]{microtype}
\UseMicrotypeSet[protrusion]{basicmath} % disable protrusion for tt fonts
}{}
\IfFileExists{parskip.sty}{%
\usepackage{parskip}
}{% else
\setlength{\parindent}{0pt}
\setlength{\parskip}{6pt plus 2pt minus 1pt}
}
\usepackage{hyperref}
\hypersetup{
            pdftitle={Improving Transparency, Falsifiability, and Rigour by Making Hypothesis Tests Machine Readable},
            pdfkeywords={hypothesis testing, machine readability, metadata, scholarly communication},
            pdfborder={0 0 0},
            breaklinks=true}
\urlstyle{same}  % don't use monospace font for urls
\usepackage{graphicx,grffile}
\makeatletter
\def\maxwidth{\ifdim\Gin@nat@width>\linewidth\linewidth\else\Gin@nat@width\fi}
\def\maxheight{\ifdim\Gin@nat@height>\textheight\textheight\else\Gin@nat@height\fi}
\makeatother
% Scale images if necessary, so that they will not overflow the page
% margins by default, and it is still possible to overwrite the defaults
% using explicit options in \includegraphics[width, height, ...]{}
\setkeys{Gin}{width=\maxwidth,height=\maxheight,keepaspectratio}
\setlength{\emergencystretch}{3em}  % prevent overfull lines
\providecommand{\tightlist}{%
  \setlength{\itemsep}{0pt}\setlength{\parskip}{0pt}}
\setcounter{secnumdepth}{0}

% set default figure placement to htbp
\makeatletter
\def\fps@figure{htbp}
\makeatother

% Manuscript styling
\usepackage{upgreek}
\captionsetup{font=singlespacing,justification=justified}

% Table formatting
\usepackage{longtable}
\usepackage{lscape}
% \usepackage[counterclockwise]{rotating}   % Landscape page setup for large tables
\usepackage{multirow}		% Table styling
\usepackage{tabularx}		% Control Column width
\usepackage[flushleft]{threeparttable}	% Allows for three part tables with a specified notes section
\usepackage{threeparttablex}            % Lets threeparttable work with longtable

% Create new environments so endfloat can handle them
% \newenvironment{ltable}
%   {\begin{landscape}\begin{center}\begin{threeparttable}}
%   {\end{threeparttable}\end{center}\end{landscape}}
\newenvironment{lltable}{\begin{landscape}\begin{center}\begin{ThreePartTable}}{\end{ThreePartTable}\end{center}\end{landscape}}

% Enables adjusting longtable caption width to table width
% Solution found at http://golatex.de/longtable-mit-caption-so-breit-wie-die-tabelle-t15767.html
\makeatletter
\newcommand\LastLTentrywidth{1em}
\newlength\longtablewidth
\setlength{\longtablewidth}{1in}
\newcommand{\getlongtablewidth}{\begingroup \ifcsname LT@\roman{LT@tables}\endcsname \global\longtablewidth=0pt \renewcommand{\LT@entry}[2]{\global\advance\longtablewidth by ##2\relax\gdef\LastLTentrywidth{##2}}\@nameuse{LT@\roman{LT@tables}} \fi \endgroup}

% \setlength{\parindent}{0.5in}
% \setlength{\parskip}{0pt plus 0pt minus 0pt}

% \usepackage{etoolbox}
\makeatletter
\patchcmd{\HyOrg@maketitle}
  {\section{\normalfont\normalsize\abstractname}}
  {\section*{\normalfont\normalsize\abstractname}}
  {}{\typeout{Failed to patch abstract.}}
\makeatother
\shorttitle{Machine Readable Hypotheses}
\author{Daniël Lakens\textsuperscript{1}\ \& Lisa M. DeBruine\textsuperscript{2}}
\affiliation{
\vspace{0.5cm}
\textsuperscript{1} School of Innovation Sciences, Eindhoven University of Technology\\\textsuperscript{2} Institute of Neuroscience and Psychology, University of Glasgow}
\authornote{Both authors contributed equally to the manuscript. First authorship was determined based on a Great League trainer battle between the authors in Pokemon Go.

Lisa DeBruine ![](images/orcid.png) https://orcid.org/0000-0002-7523-5539. 

Daniël Lakens ![](images/orcid.png) https://orcid.org/0000-0002-0247-239X.


Correspondence concerning this article should be addressed to Daniël Lakens, ATLAS 9.402, 5600 MB, Eindhoven, The Netherlands. E-mail: D.Lakens@tue.nl}
\note{ }
\keywords{hypothesis testing, machine readability, metadata, scholarly communication\newline\indent Word count: }
\usepackage{lineno}

\linenumbers
\usepackage{csquotes}
\ifnum 0\ifxetex 1\fi\ifluatex 1\fi=0 % if pdftex
  \usepackage[shorthands=off,main=english]{babel}
\else
  % load polyglossia as late as possible as it *could* call bidi if RTL lang (e.g. Hebrew or Arabic)
  \usepackage{polyglossia}
  \setmainlanguage[]{english}
\fi

\title{Improving Transparency, Falsifiability, and Rigour by Making Hypothesis Tests Machine Readable}

\date{}

\abstract{
Making scientific information machine-readable greatly facilitates its re-use. Many scientific articles have the goal to test a hypothesis, and making the tests of statistical predictions easier to find and access could be very beneficial. We propose an approach that can be used to make hypothesis tests machine readable. We believe there are two benefits to specifying a hypothesis test in a way that a computer can evaluate whether the statistical prediction is corroborated or not. First, hypothesis test will become more transparent, falsifiable, and rigorous. Second, scientists will benefit if information related to hypothesis tests in scientific articles is easily findable and re-usable, for example when performing meta-analyses, during peer review, and when examining meta-scientific research questions. We examine what a machine readable hypothesis test should looks like, and demonstrate the feasibility of machine readable hypothesis tests in a real-life example.
}

\begin{document}
\maketitle

In many scientific fields researchers rely on hypothesis tests to determine whether empirical observations corroborate predictions. In a well-specified hypothesis test, a theoretical hypothesis is used to derive predictions, which are operationalized when designing a specific study, and translated into a testable statistical hypothesis. Data is collected, and the statistical hypothesis is corroborated or not. Although this process sounds relatively straightforward, hypothesis tests are performed rather poorly in practice. First, statistical hypotheses are stated verbally, but these verbal descriptions rarely sufficiently constrain flexibility in the data analysis. Second, there is a lack of transparency about which statistical tests in the results section are related to the predictions in the introduction section, and which pattern of results should be observed to conclude that a prediction is corroborated. Finally, researchers typically only implicitly specify what would lead them to act as if their prediction is confirmed (i.e., typically a \emph{p}-value smaller than 0.05), and rarely specify what would lead them to act as if their prediction is falsified. By contrast, a well-specified hypothesis test states the statistical hypothesis for each prediction in a way that eliminates flexible implementations, clearly links predictions derived from the theoretical hypothesis to statistical tests, and gives unambiguous criteria to conclude the prediction is corroborated, falsified, or that the results are inconclusive.

We propose that the gold standard for well-specified hypothesis tests should be a statistical prediction that is machine readable. This means that a computer can evaluate whether a statistical prediction is corroborated (or not) based on clearly articulated evaluation criteria and the observed data. Computers do not handle ambiguity well, and making a hypothesis test machine readable guarantees that it is specified precisely. In this manuscript we demonstrate how hypothesis tests can be made machine readable. We believe that there are two broad arguments for a move to machine readable hypothesis tests. The first argument is that by specifying hypothesis tests in a format that can be read and evaluated by a machine, tests of predictions and the conclusions derived from these tests will become transparent, statistically falsifiable, and rigorous. This will help to alleviate poor practices in how scientists currently test hypotheses. The second argument is that the benefits of making data FAIR (findable, accessible, interoperable, and reusable) also apply to statistical predictions. If all aspects required to evaluate the test of a statistical prediction are machine-readable, we can easily reuse this information (e.g., when performing meta-analyses), and find and access this information (e.g., to answer meta-scientific questions about the proportion of statistical results in the scientific literature that corroborate the prediction).

\hypertarget{poor-practices-when-testing-predictions}{%
\subsection{Poor practices when testing predictions}\label{poor-practices-when-testing-predictions}}

As a concrete example of a typical hypothesis test in the published literature, DeBruine (2002) posited the theoretical prediction that people would exhibit higher levels of prosocial behavior towards those who physically resemble them, which follows from the idea that actions are influenced by an implicit evaluation of relatedness based on phenotypic similarity. Physical resemblance was manipulated by morphing face photographs with either the participant's own face (self morphs) or another person's face (other morphs). There were two versions of this manipulation: faces were morphed in shape only (n = 11) or in both shape and color (n = 13). Prosocial behavior was measured as the choice to trust or reciprocate trust in a monetary trust game where the first player could decide whether to trust the second player to split money and the second player, if trusted, could decide whether to reciprocate this trust by splitting the money equally or selfishly. The theoretical hypothesis was operationalized, and the operationalized prediction stated that people playing a trust game would trust and reciprocate more when playing with a person who was represented by a self morph than by an other morph. The statistical prediction was tested by counting the number of trusting and reciprocating responses participants made to self and other morphs and then performing a \emph{t}-test on these counts, separately analyzed for the shape morphs and the shape-colour morphs. The statistical results indicated that participants made more trust responses to self morphs than to other morphs for both morph types. However, there were no differences in how often they reciprocated their partners' trust. The conclusion drawn from this study was that these results show that facial resemblance can increase prosocial behaviour. It was noted that the fact that an effect was observed for the trust measure, but not for the reciprocation measure, could perhaps be explained by the different pay-off structures in the two-person bargaining game.

The first problem we can identify in this example is that it is not clear whether the operationalized prediction was confirmed if an effect was observed on both the trust measure and the reciprocation measure, or either of the two measures. From the conclusion the author draws we can infer that statistical prediction would be considered corroborated if the morphing manipulation had an effect on either the trust measure, or the reciprocation measure, or both. However, it is never clearly specified before the conclusion is reported which pattern of results would corroborate the prediction (e.g., if either, or both, of the tests were predicted to be statistically significant).

The second problem is that is is not clearly specified what would corroborate the hypothesis and what would statistically falsify the hypothesis. Although it is never explicitly stated, we can infer that the prediction would be corroborated when either of the two tests is significant at an alpha level of 0.05, without correcting for multiple comparisons. Furthermore, we can infer that a non-significant \emph{p}-value is interpreted as the absence of any meaningful effect (even though this is a formally incorrect interpretation of a null hypothesis test). The third problem is that there is a range of options when analyzing the data (e.g., pooling the two types of morphs in one analysis, or reporting two separate analyses by morph version). As is often the case the testing statistical predictions, no unique analysis strategy follows unequivocally from the introduction and methods section, which can lead to flexibility in the data analysis.

\hypertarget{what-does-a-formalized-test-of-a-prediction-look-like}{%
\subsection{What Does a Formalized Test of a Prediction Look Like?}\label{what-does-a-formalized-test-of-a-prediction-look-like}}

If we want to make hypothesis tests machine readable, we need to capture all essential aspects of a hypothesis test in a machine readable data structure. A hypothesis test is a methodological procedure to evaluate a prediction that can be described on a conceptual level (e.g., people exhibit higher levels of prosocial behavior towards those who physically resemble them), an operationalized level (e.g., people playing a trust game make more trusting decisions when the person they play against is a self morph versus an other morph), and a statistical level (e.g., the average number of trust moves is statistically larger for games against self morphs than against other morphs in a dependent \emph{t}-test).

We distinguish the following components of a statistical prediction: 1) a statistical test, 2) a test result, and 3) one or more criterion values that the test result is compared to. For example, our statistical prediction might be that we will observe a positive difference in the means between two measurements, which will be examined in a dependent \emph{t}-test, from which we will determine the lower and upper 97.5\% confidence interval around the mean difference, which we will compare against a value of 0. Statistical hypotheses are probabilistic, and probabilistic hypotheses can be made falsifiable \enquote{by specifying certain rejection rules which may render statistically interpreted evidence \enquote{inconsistent} with the probabilistic theory} (Lakatos, 1978). A hypothesis test thus requires researchers to specify when the observed results of a statistical test will lead them to act as if their prediction is consistent with the data, inconsistent with the data, or inconclusive (Dienes, 2019; Neyman, Pearson, \& Pearson, 1933).

As highlighted above, one limitation of current practice is that researchers often do not explicitly state what would corroborate or falsify their prediction. To be able to unambiguously evaluate a hypothesis, researchers need to specify evaluation rules for when they will interpret statistical test results as corroborating a prediction, falsifying of a prediction, or when a result will be treated as inconclusive. For example, a prediction might be consistent with a main effect of A, or an ordinal interaction between A and B, but inconsistent with a disordinal interaction between A and B. However, the full pattern of possible results that would corroborate or falsify a prediction is seldom explicit. There are different statistical approaches that can be used to statistically conclude a prediction is falsified, and one should always think meta-analytically, and keep random variation in mind. In practice, corroborating or falsifying a statistical prediction is rarely sufficient to draw strong conclusions about a theoretical prediction (Lakatos, 1978).

One approach, known as equivalence testing (Lakens, Scheel, \& Isager, 2018), requires researchers to specify a smallest effect size of interest, and tests if the presence of an effect that is large enough to be deemed interesting can be statistically rejected. Continuing our example, we might conclude our prediction is corroborated when we can statistically conclude the observed mean difference is greater than zero, and not statistically smaller than the smallest effect size we care about. The prediction would be falsified if the effect is statistically smaller than the smallest effect size of interest, and inconclusive if we can neither conclude the effect is statistically greater than zero, nor statistically smaller than the smallest effect size we care about. If our statistical test is a dependent \emph{t}-test, our test result is the upper and lower bound of a 97.5\% confidence interval (i.e., a hypothesis test with a 2.5\% alpha level), and our smallest effect size of interest is 0.2, we can conclude that we have corroborated our prediction if the lower bound of our 97.5\% CI is larger than 0 and the upper bound is not smaller than 0.2. Our prediction is falsified if the upper bound of our 97.5\% CI is smaller than 0.2, and our data is inconclusive in all other situations.

Another approach compares the likelihood of the observed data under null and theoretical models (Dienes, 2019). If the ratio of these likelihoods (the Bayes factor) is greater than some specified criterion (e.g., 6, used by the journal \emph{Cortex}), this is interpreted as evidence for the hypothetical model. If the Bayes factor is less than the reciprocal of the criterion (e.g., 1/6), this is interpreted as evidence for the null model. All other values are interpreted as inconclusive evidence.

\hypertarget{computationally-evaluating-hypotheses}{%
\subsection{Computationally Evaluating Hypotheses}\label{computationally-evaluating-hypotheses}}

If a prediction is machine readable, it is possible to automatically determine if a prediction is corroborated by the data. We envision machine readable hypothesis tests as part of a completely reproducible workflow. Computer scripts will load the raw data, and if needed create the analytic data from the raw data (e.g., outlier removal, transformations, computing sum scores according to pre-specified rules). The statistical test is automatically run on the analytic data, and the relevant test statistics are retrieved. These test statistics are compared against pre-specified criteria, based on decision rules that evaluate whether the prediction is corroborated, falsified, or inconclusive. All the information that is required to perform these operations is stored in a structured meta-data file.

Box 1 provides a concrete example of a machine-readable statistical prediction for the study by DeBruine (2002) described above. It is written using JSON, which is an open-standard file format that can be used to transmit data. Because it is an open-standard file format, it can easily be converted into any other data file format such as YAML or JATS, which in essence are all nested lists. It can also be converted to a human-readable report, summarising the study with verbal descriptions and a list containing the conclusion for each statistical prediction.

The top level list contains components describing different aspects of the study, such as authors, hypotheses, materials, methods, data, and analyses. In the future we might be able to describe all information in a scientific article that we would like to be able to retrieve, but here we will focus on the aspects of the study that are required to make statistical predictions machine readable. To achieve this, we need a meta-data file that specifies the hypotheses, the analyses, and the evaluation criteria for each prediction.

A study could contain multiple hypotheses, but our example contains only one. Each \texttt{hypothesis} (Box 1, lines 3-50) consists of a verbal human-readable description (5), one or more \texttt{criteria} (6-39) to evaluate analysis results, and rules to determine if the results \texttt{corroborate} (40-44) or \texttt{falsify} (45-49) the hypothesis that is automatically generated if the data is available. Criteria need an \texttt{id} to be able to reference them in the evaluations. An \texttt{operator} and a \texttt{comparator} are provided for each criterion to specify the method of comparison (e.g., \textgreater{}, \textless{}, =, !=) and the comparison value (e.g., 0). For example, the first criterion (7-14) specifies that if the statistical result \enquote{conf.int{[}1{]}} from \enquote{trust\_analysis} is \enquote{\textgreater{}} than \enquote{0} then the criterion \enquote{trust\_lowbound} evaluates to \enquote{true}. In other words, if we can statistically reject the null hypothesis (because the lower bound of the confidence interval does not overlap with 0), this criterion of our statistical prediction is corroborated.

The \texttt{corroborate} (40-44) and \texttt{falsify} (45-49) sub-components describe rules to determine corroborate or falsification from the criteria conclusions, and consist of three elements. The \texttt{description} element contains verbal descriptions of the decision rules for concluding the hypothesis is corroborated or falsified. The \texttt{evaluation} element contains a logical version referencing the criteria IDs. For example, \enquote{\texttt{(trust\_lowbound\ \&\ trust\_highbound)\ \textbar{}\ (recip\_lowbound\ \&\ recip\_highbound)}} (42) means that the corroboration conclusion (43) will be set to \enquote{true} if the first two criteria are both true, or if the last two criteria are both true, while \enquote{\texttt{!trust\_highbound\ \&\ !recip\_highbound}} (47) means that the falsify conclusion (48) will be set to \enquote{true} if both of these criteria are false.

Each analysis is specified in the \texttt{analysis} component (52-81). An analysis consists of an \texttt{id} to reference the statistical test when evaluating the criteria, a reference to the function or script used to run the analysis (\texttt{func}), a list of parameters that need to be specified in the function (\texttt{params}), and a list of named \texttt{results} to be referenced in the criteria. Each analysis can also contain additional information, such as the software used to perform the analysis. The example below specifies two \emph{t}-tests, using the \texttt{t.test} function in R (version 3.6.0). For each \emph{t}-test, we need to specify values for \enquote{x} and \enquote{y} (columns in a dataset called \enquote{kin}), \enquote{paired} (the type of \emph{t}-test), and \enquote{conf.level} (to change the default value for this test to .975).

Each dataset can be specified in the \texttt{data} component (82-112). A dataset consists of an \texttt{id} to reference the dataset in analyses and information about how to obtain the data (e.g., \texttt{doi}, \texttt{url}). The \texttt{codebook} (87-104) contains descriptions of each column, and it is even possible to include the analytic data \texttt{values} (105-110) in this component. Below, we present a simple version of a codebook, but the descriptors for each column can be arbitrarily detailed and automatically extracted from existing data structures, such as SPSS files. For software that helps researchers to share machine-readable codebooks, see Arslan (2019).
\newpage

\hypertarget{box-1.-example-json-file-illustrating-a-machine-readable-statistical-prediction.}{%
\subsubsection{Box 1. Example JSON file illustrating a machine-readable statistical prediction.}\label{box-1.-example-json-file-illustrating-a-machine-readable-statistical-prediction.}}

\begin{verbatim}
  1. {
  2.   "name": "Kinship and Prosocial Behaviour",
  3.   "hypotheses": [
  4.     {
  5.       "description": "Cues of kinship will increase prosocial behaviour.
              Cues of kinship will be manipulated by morphed facial self-
              resemblance. Prosocial behaviour will be measured by responses in
              the trust game. The prediction is that the number of trusting AND/
              OR reciprocating moves will be greater to self morphs than to other
              morphs.",
  6.       "criteria": [
  7.         {
  8.           "id": "trust_lowbound",
  9.           "analysis_id": "trust_analysis",
 10.           "result": "conf.int[1]",
 11.           "operator": ">",
 12.           "comparator": 0,
 13.           "conclusion": true
 14.         },
 15.         {
 16.           "id": "trust_highbound",
 17.           "analysis_id": "trust_analysis",
 18.           "result": "conf.int[2]",
 19.           "operator": ">",
 20.           "comparator": 0.2,
 21.           "conclusion": true
 22.         },
 23.         {
 24.           "id": "recip_lowbound",
 25.           "analysis_id": "recip_analysis",
 26.           "result": "conf.int[1]",
 27.           "operator": ">",
 28.           "comparator": 0,
 29.           "conclusion": false
 30.         },
 31.         {
 32.           "id": "recip_highbound",
 33.           "analysis_id": "recip_analysis",
 34.           "result": "conf.int[2]",
 35.           "operator": ">",
 36.           "comparator": 0.2,
 37.           "conclusion": false
 38.         }
 39.       ],
 40.       "corroboration": {
 41.         "description": "The hypothesis is corroborated if the 97.5% CI lower
              bound is greater than 0 and the 97.5% CI upper bound is greater than
              0.2 (the SESOI) for either the trust or reciprocation moves.",
 42.         "evaluation": "(trust_lowbound & trust_highbound) | (recip_lowbound &
              recip_highbound)",
 43.         "conclusion": true
 44.       },
 45.       "falsify": {
 46.         "description": "The hypothesis is falsified if the 97.5% CI
              upper bound is smaller than 0.2 (the SESOI) for both trust and
              reciprocation.",
 47.         "evaluation": "!trust_highbound & !recip_highbound",
 48.         "conclusion": false
 49.       }
 50.     }
 51.   ],
 52.   "analyses": [
 53.     {
 54.       "id": "trust_analysis",
 55.       "software": "R version 3.6.0 (2019-04-26)",
 56.       "func": "t.test",
 57.       "params": {
 58.         "x": "kin$trust_self",
 59.         "y": "kin$trust_non",
 60.         "paired": true,
 61.         "conf.level": 0.975
 62.       },
 63.       "results": {
 64.         "conf.int": [0.02, 0.98]
 65.       }
 66.     },
 67.     {
 68.       "id": "recip_analysis",
 69.       "software": "R version 3.6.0 (2019-04-26)",
 70.       "func": "t.test",
 71.       "params": {
 72.         "x": "kin$recip_self",
 73.         "y": "kin$recip_non",
 74.         "paired": true,
 75.         "conf.level": 0.975
 76.       },
 77.       "results": {
 78.         "conf.int": [-0.51 0.43]
 79.       }
 80.     }
 81.   ],
 82.   "data": [
 83.     {
 84.       "id": "kin",
 85.       "doi": "10.17605/OSF.IO/F7QWS",
 86.       "url": "https://osf.io/ewfhs/",
 87.       "codebook": [
 88.         {
 89.           "name": "trust_self",
 90.           "description": "Number of trusting moves towards self-morphs"
 91.         },
 92.         {
 93.           "name": "trust_other",
 94.           "description": "Number of trusting moves towards self-morphs"
 95.         },
 96.         {
 97.           "name": "recip_self",
 98.           "description": "Number of reciprocating moves towards other-morphs"
 99.         },
100.         {
101.           "name": "recip_other",
102.           "description": "Number of reciprocating moves towards other-morphs"
103.         }
104.       ],
105.       "values": {
106.         "trust_self": [1,2,2,1,1,1,1,1,2,0,2,0,1,2,2,3,2,2,1,1,2,0,0,1],
107.         "trust_other": [1,2,2,0,1,0,0,0,1,0,1,0,1,1,1,0,1,2,2,0,0,0,2,1],
108.         "recip_self": [0,1,3,2,1,1,1,3,3,2,3,1,1,2,3,3,3,1,1,1,3,0,3,1],
109.         "recip_other": [1,1,2,2,3,2,1,3,3,1,3,0,1,3,3,3,3,0,3,0,1,0,3,2]
110.       }
111.     }
112.   ]
113. }
\end{verbatim}

The statistical prediction can be evaluated automatically, without human intervention, as long as the prediction is specified in a machine readable format. Based on the information specified in the analyses and criteria sections of the meta-data, a computer script can read in the analytic data, perform each analysis specified in the JSON file, and store the results. For example, the \enquote{trust\_analysis} can be performed by running the analysis \texttt{t.test(x\ =\ kin\textbackslash{}\$trust\_self,\ y\ =\ kin\$trust\_other,\ paired\ =\ TRUE,\ conf.level\ =\ .975)} after the data is loaded as an object named \enquote{kin}. We store the result of this analysis (e.g., the t.test function in R returns a list of named numbers, including \enquote{conf.int}: {[}0.02, 0.98{]} in line 63). Based on the results of the analyses, the code can compare the results against the criteria. For example, because the first number in the \enquote{conf.int} result (0.02) is larger (\enquote{\textgreater{}}) than zero (\enquote{0}), we can store the conclusion that this criterion is \enquote{true} (line 13). After the code has drawn conclusions about whether each criterion is met or not, based on the results of the analyses, the evaluation rules can be used to determine whether the prediction is corroborated, falsified, or neither (and thus the results are inconclusive). For the prediction to be corroborated, the criteria for trust\_lowbound and trust\_highbound have to be met, and/or the criteria for recip\_lowbound and recip\_highbound have to be met. Since the conclusions for trust\_lowbound and trust\_highbound are both true, the prediction is corroborated, and because it is not true that both upper bounds for the confidence interval are smaller than 0.2, the prediction is not falsified. The overall conclusion is therefore that our statistical prediction is corroborated

\hypertarget{benefits-of-machine-readability}{%
\subsection{Benefits of Machine Readability}\label{benefits-of-machine-readability}}

We believe the benefits of making statistical predictions machine readable are worth the effort. First, machine-readable hypotheses completely remove ambiguity about what researchers predict and which criteria must be met to conclude a statistical prediction hypothesis is corroborated. Although ambiguity can be removed in other ways, the formalized approach as illustrated in Box 1 is perfectly suited for this goal. Predictions are explicitly linked to the tests that are performed to evalaute if the prediction is corroborated or not. The exact test is specified, which prevents flexibility in the data analysis. This method prevents researchers who will replicate the study from having to infer the criteria for corroboration or falsification from the observed pattern of results and the conclusion. If a researchers feels the statistical prediction can be tested in different ways, each equally reasonable, a range of sensitivity analyses across which the prediction should hold can be specified.

While this method obviously cannot ensure that the logic behind predictions or criteria for corroboration or falsification are correct, the process of writing a machine-readable statistical prediction has a secondary benefit of providing a framework for thinking through the logic of a prediction. If one finds it impossible to specify the ranges of results that will corroborate or falsify a prediction, it is likely that the hypothesis is not yet well-specified enough for confirmatory hypothesis testing. Hypothesis tests are an extremely formalized procedure to test predictions. Making this clear might make researchers realize they are not yet ready to test a hypothesis, and remind them of the importance and value of exploratory research.

Another benefit of making statistical hypotheses machine readable is that many important aspects of the hypothesis test become accessible, findable, and usable. This will benefit researchers in the future. We can imagine a utopian future where meta-data files such as the example in Box 1 are accessible by browsing to a website that consists of the DOI, appended by /meta (e.g., \url{https://doi.org/10.1098/rspb.2002.2034/meta/}). Researchers can access these files to load all the information that is available about statistical predictions. When a completely reproducible workflow is used, and data can be accessed, the meta-data file should be sufficient to easily calculate or access effect sizes from the performed statistical tests for meta-analyses, and answer meta-scientific questions about, for example, the percentage of statistical predictions that are corroborated in a research area.

One more immediate use case for machine readable hypothesis tests is the Registered Report publication format (Chambers, 2019). Registered Reports require researchers to clearly specify their statistical prediction, and are developed to reduce flexibility in the statistical analyses. After Stage 1 review based on the introduction, methods, and analysis plan, researchers can receive an \enquote{in principle acceptance}. They then collect the data, and submit a Stage 2 Registered Report that includes the results and conclusion. Reviewers need to evaluate whether the analyses were conducted exactly as planned and whether conclusions follow from the predictions. This aspect of the peer review process is labor intensive and often still somewhat ambiguous, but could be automated if the statistical predictions were machine readable.

\hypertarget{conclusions}{%
\subsection{Conclusions}\label{conclusions}}

Technological innovation makes it possible to communicate scientific findings in digital formats that allow for much easier re-use of scientific information contained in these digital files compared to traditional journal articles. As we move towards a time where researchers are expected to share their data in a way that is FAIR (findable, accessible, interoperable, and reusable), we believe it is feasible and beneficial to make statistical predictions machine readable as well. We see machine readable hypothesis tests as a logical development, with immediate benefits for the rigour of hypothesis tests. Increasing the accessibility of essential information related to hypothesis tests in scientific paper will also facilitate peer review, especially of Registered Reports, and facilitate meta-scientific research. Making statistical predictions machine readable will be an important next step towards a scientific literature that can be accessed not just visually, but also computationally.

\newpage

\hypertarget{acknowledgements}{%
\subsubsection{Acknowledgements}\label{acknowledgements}}

We would like to thank Leo Tiokhin and Peder Isager for feedback on an earlier draft of this manuscript, and attendants of a hackathon at the Society for the Improvement of Psychological Science for their enthusiastic reception of the ideas behind machine readable hypotheses.

\hypertarget{author-contributions}{%
\subsubsection{Author Contributions}\label{author-contributions}}

Both authors conceptualized the main idea, LMD wrote the Scienceverse software, and both authors wrote and revised this manuscript.

\hypertarget{declaration-of-conflicting-interests}{%
\subsubsection{Declaration of Conflicting Interests}\label{declaration-of-conflicting-interests}}

The author(s) declared that there were no conflicts of interest with respect to the authorship or the publication of this article.

\hypertarget{funding}{%
\subsubsection{Funding}\label{funding}}

LMD is supported by European Research Council grant \#647910. DL is funded by VIDI Grant 452-17-013 from the Netherlands Organisation for Scientific Research.

\newpage

\hypertarget{references}{%
\section{References}\label{references}}

\begingroup
\setlength{\parindent}{-0.5in}
\setlength{\leftskip}{0.5in}

\hypertarget{refs}{}
\leavevmode\hypertarget{ref-arslan2019}{}%
Arslan, R. C. (2019). How to automatically document data with the codebook package to facilitate data reuse. \emph{Advances in Methods and Practices in Psychological Science}, \emph{2}(2), 169--187. \url{https://doi.org/10.1177/2515245919838783}

\leavevmode\hypertarget{ref-chambers2019}{}%
Chambers, C. (2019). What's next for registered reports? \emph{Nature}, \emph{573}, 187--189. \url{https://doi.org/10.1038/d41586-019-02674-6}

\leavevmode\hypertarget{ref-debruine2002}{}%
DeBruine, L. M. (2002). Facial resemblance enhances trust. \emph{Proceedings of the Royal Society of London. Series B: Biological Sciences}, \emph{269}, 1307--1312. \url{https://doi.org/10.1098/rspb.2002.2034}

\leavevmode\hypertarget{ref-dienes2019}{}%
Dienes, Z. (2019). How do I know what my theory predicts? \emph{Advances in Methods and Practices in Psychological Science}, \emph{2}(4), 364--377. \url{https://doi.org/10.1177/2515245919876960}

\leavevmode\hypertarget{ref-lakatos1978}{}%
Lakatos, I. (1978). \emph{The methodology of scientific research programmes: Volume 1: Philosophical papers.} Cambridge University Press.

\leavevmode\hypertarget{ref-lakens2018}{}%
Lakens, D., Scheel, A. M., \& Isager, P. M. (2018). Equivalence testing for psychological research: A tutorial. \emph{Advances in Methods and Practices in Psychological Science}, \emph{1}(2), 259--269. \url{https://doi.org/10.1177/2515245918770963}

\leavevmode\hypertarget{ref-neyman1933}{}%
Neyman, J., Pearson, E. S., \& Pearson, K. (1933). IX. On the problem of the most efficient tests of statistical hypotheses. \emph{Philosophical Transactions of the Royal Society of London. Series A, Containing Papers of a Mathematical or Physical Character}, \emph{231}(694-706), 289--337. \url{https://doi.org/10.1098/rsta.1933.0009}

\endgroup

\end{document}
